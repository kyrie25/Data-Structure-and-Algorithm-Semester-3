\section{Giới thiệu}

\subsection{Nội dung}
Trong lĩnh vực công nghệ thông tin, các thuật toán sắp xếp (\textit{sort}) đóng vai trò quan trọng trong việc tổ chức và xử lý dữ liệu. Từ việc sắp xếp danh sách học sinh theo điểm số, đến việc sắp xếp các giao dịch tài chính theo thời gian, các thuật toán sắp xếp được áp dụng rộng rãi trong nhiều tình huống thực tế. Việc hiểu rõ và cải tiến các thuật toán này không chỉ giúp nâng cao hiệu suất của hệ thống máy tính mà còn mở ra những ứng dụng mới trong nhiều lĩnh vực khác nhau.
\subsection{Mục đích}
Nghiên cứu này nhằm mục đích so sánh hiệu suất của các thuật toán sort phổ biến như \textbf{Bubble sort, Counting sort, Flash sort, Heap sort, Insertion sort, Merge sort, Quick sort, Radix sort, Selection sort, Shaker sort, Shell sort} trong các tình huống khác nhau. Đồng thời, nghiên cứu cũng tìm hiểu các yếu tố ảnh hưởng đến hiệu suất của các thuật toán này, từ đó đề xuất các cải tiến để tối ưu hóa hiệu suất.\\
Nghiên cứu này đóng góp vào cộng đồng khoa học bằng việc cung cấp một phân tích toàn diện và cập nhật về hiệu suất của các thuật toán \textit{sort}. Ngoài ra, nghiên cứu cũng đề xuất các cải tiến mới để tối ưu hóa hiệu suất của các thuật toán này, đặc biệt trong môi trường tính toán hiện đại với nhiều yếu tố phức tạp.

\subsection{Phương pháp nghiên cứu}

\subsubsection{Phân tích lý thuyết}
Phương pháp này tập trung vào việc phân tích các thuộc tính của thuật toán từ góc độ lý thuyết:

\begin{itemize}
    \item [\textbf{--}] \textbf{So sánh độ phức tạp thời gian (\textit{Time Complexity}):}
    \begin{itemize}
        \item [$\bullet$]Xác định và chứng minh độ phức tạp thời gian của từng thuật toán dựa trên số lượng phép so sánh.
        \item [$\bullet$]Phân tích các trường hợp khác nhau:
        \begin{itemize}
            \item [$\bullet$] Tốt nhất (\textit{Best Case}): Khi đầu vào phù hợp tối ưu.
            \item [$\bullet$] Trung bình (\textit{Average Case}): Khi đầu vào ngẫu nhiên.
            \item [$\bullet$] Tệ nhất (\textit{Worst Case}): Khi đầu vào không phù hợp.
        \end{itemize}
    \end{itemize}
    
    \item [\textbf{--}] \textbf{So sánh độ phức tạp không gian (\textit{Space Complexity}):}
    \begin{itemize}
        \item [$\bullet$] Phân tích lượng bộ nhớ cần thiết cho từng thuật toán.
        \item [$\bullet$] Xem xét liệu thuật toán có sử dụng bộ nhớ bổ sung (\textit{In-place sorting}) hay cần bộ nhớ phụ.
    \end{itemize}
    
    \item [\textbf{--}] \textbf{Phân tích đặc điểm:}
    \begin{itemize}
        \item [$\bullet$] Đặc điểm ổn định (\textit{Stable}) hay không ổn định (\textit{Unstable}).
        \item [$\bullet$] Tích hợp với dữ liệu lớn, dữ liệu ngẫu nhiên, dữ liệu đã sắp xếp, dữ liệu sắp xếp ngược thứ tự, hoặc dữ liệu gần được sắp xếp.
    \end{itemize}
\end{itemize}

\subsubsection{Thực nghiệm}
Phương pháp thực nghiệm giúp kiểm tra hiệu năng thực tế của thuật toán:

\begin{itemize}
    \item [\textbf{--}] \textbf{Chuẩn bị dữ liệu đầu vào:}
    \begin{itemize}
    \item [$\bullet$] Khảo sát các thuật toán sắp xếp trên các kích thước mảng khác nhau: 10000, 30000, 50000, 100000, 300000 và 500000.
    \item [$\bullet$] Khảo sát các thuật toán sắp xếp trên các kiểu thứ tự khác nhau của mảng:
        \begin{itemize}
            \item [$\bullet$] Mảng đã sắp xếp.
            \item [$\bullet$] Mảng gần được sắp xếp hoàn chỉnh.
            \item [$\bullet$] Mảng đã sắp xếp, nhưng theo thứ tự ngược (sắp xếp ngược).
            \item [$\bullet$] Mảng có thứ tự ngẫu nhiên.
        \end{itemize}
    \end{itemize}
    \item [\textbf{--}] \textbf{Các chỉ số hiệu năng cần ghi nhận:}
    \begin{itemize}
        \item [$\bullet$] Thời gian thực thi: Đo thời gian cần thiết để sắp xếp.
        \item [$\bullet$] Số phép so sánh: Đếm số lần các phần tử được so sánh.
    \end{itemize}
    \item [\textbf{--}] \textbf{Công cụ hỗ trợ thực nghiệm:}
    \begin{itemize}
        \item [$\bullet$]Viết các đoạn mã nguồn bằng \textit{C/C++}.
        \item [$\bullet$]Mã nguồn được biên dịch thành chương trình sử dụng tham số dòng lệnh (command-line argument) với các chế độ mode sau:
        \begin{itemize}
            \item [$\bullet$] \textit{Algorithm mode} (chạy 1 thuật toán).
            \item [$\bullet$] \textit{Comparison mode} (chạy 1 cặp thuật toán).
        \end{itemize}
    \end{itemize}
    \item [\textbf{--}]\textbf{So sánh kết quả:}
    \begin{itemize}
        \item [$\bullet$]Trình bày kết quả dưới dạng bảng và đồ thị.
        \item [$\bullet$]Phân tích sự khác biệt giữa lý thuyết và thực nghiệm.
    \end{itemize}
\end{itemize}

\subsubsection{Mô phỏng và trực quan hóa}
Mô phỏng và trực quan hóa giúp minh họa cách hoạt động của thuật toán, dễ hiểu hơn cho độc giả:

\begin{itemize}
    \item [\textbf{--}]\textbf{Mô phỏng:}
    \begin{itemize}
        \item [$\bullet$]Tạo các đoạn mã giả hoặc hình vẽ minh hoạt từng bước sắp xếp.
        \item [$\bullet$]Hiển thị sự thay đổi vị trí các phần tử trong mảng qua từng bước của thuật toán.
    \end{itemize}
    \item [\textbf{--}]\textbf{Trực quan hóa:}
    \begin{itemize}
        \item [$\bullet$]Vẽ biểu đồ cột minh hoạt số phép so sánh, thời gian thực thi.
        \item [$\bullet$]So sánh các thuật toán bằng biểu đồ đường.
        \item [$\bullet$] Phân tích hiệu suất thuật toán bằng phương pháp \textbf{logarithmic scale}, chia tỷ lệ dữ liệu theo logarit để làm rõ xu hướng và phạm vi rộng lớn.\cite{ibm_logarithmic_scale}
    \end{itemize}
\end{itemize}
\newpage
