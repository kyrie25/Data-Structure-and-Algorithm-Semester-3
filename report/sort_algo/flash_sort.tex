\subsubsection{Ý tưởng}
Flash Sort là một thuật toán sắp xếp dựa trên việc phân phối các phần tử trong mảng thành các nhóm (lớp), dựa trên giá trị của chúng. Các phần tử sau đó được sắp xếp cục bộ trong từng nhóm và được hợp nhất để tạo thành mảng đã sắp xếp. Thuật toán này hoạt động hiệu quả với mảng có sự phân bố đồng đều.

\subsubsection{Mã giả}

\begin{algorithm}[H]
\caption{Flash Sort}
\begin{algorithmic}[1]
\Procedure{FlashSort}{$arr, n$}
    \State \textbf{Input:} Mảng $arr$ gồm $n$ phần tử
    \State \textbf{Output:} Mảng $arr$ được sắp xếp
    
    \State \textbf{Bước 1: Phân loại (Classification)}
    \State Tìm giá trị nhỏ nhất $A_{\text{min}}$ và lớn nhất $A_{\text{max}}$ trong mảng $arr$
    \State Tính số nhóm $m \gets \lfloor 0.43 \cdot n \rfloor$
    \State Khởi tạo mảng đếm $L[1..m]$ với giá trị ban đầu là 0
    \For{$i \gets 1$ \textbf{to} $n$}
        \State Xác định nhóm $k_i \gets \lfloor m \cdot \frac{arr[i] - A_{\text{min}}}{A_{\text{max}} - A_{\text{min}}} \rfloor$
        \State Tăng $L[k_i]$ lên 1
    \EndFor
    
    \State Tính ranh giới của từng nhóm
    \For{$j \gets 2$ \textbf{to} $m$}
        \State $L[j] \gets L[j] + L[j-1]$
    \EndFor
    
    \State \textbf{Bước 2: Phân phối (Permutation)}
    \State Hoán đổi các phần tử để phân phối chúng vào đúng nhóm
    \State Duyệt từng phần tử, đưa phần tử vào vị trí thích hợp dựa trên nhóm $k_i$
    
    \State \textbf{Bước 3: Sắp xếp từng nhóm (Sorting)}
    \For{$j \gets 1$ \textbf{to} $m$}
        \State Sử dụng thuật toán sắp xếp cục bộ (e.g., \textit{Insertion Sort}) để sắp xếp các phần tử trong nhóm $j$
    \EndFor
\EndProcedure
\end{algorithmic}
\end{algorithm}