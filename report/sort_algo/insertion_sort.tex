\subsubsection{Ý tưởng}
Insertion sort là một thuật toán sắp xếp đơn giản hoạt động bằng cách lặp lại việc chèn từng phần tử của một danh sách chưa được sắp xếp vào đúng vị trí của nó trong một phần đã được sắp xếp của danh sách. Nó giống như việc sắp xếp các lá bài trong tay bạn. Bạn chia các lá bài thành hai nhóm: các lá bài đã được sắp xếp và các lá bài chưa được sắp xếp. Sau đó, bạn chọn một lá bài từ nhóm chưa được sắp xếp và đặt nó vào đúng vị trí trong nhóm đã được sắp xếp.
\subsubsection{Mã giả}

\begin{algorithm}[H]
\caption{Insertion sort}
\begin{algorithmic}[1]
\Procedure{Insertion sort}{$arr, n$}
    \State \textbf{Input:} Mảng $arr$ gồm $n$ phần tử
    \State \textbf{Output:} Mảng $arr$ được sắp xếp
    
    \For{$i \gets 1$ to $n-1$}
    \State $key \gets arr[i]$
    \State $j \gets i - 1$
    \While{$j \geq 0$ \textbf{and} $arr[j] > key$}
        \State $arr[j+1] \gets arr[j]$
        \State $j \gets j - 1$
    \EndWhile
    \State $arr[j+1] \gets key$
\EndFor
\EndProcedure
\end{algorithmic}
\end{algorithm}

\subsubsection{Ví dụ}
Giả sử chúng ta có mảng ban đầu: $[42, 17, 93, 58, 21, 76, 34]$. Dưới đây là các bước thực hiện Insertion sort minh họa bằng hình ảnh:


\begin{figure}[H]
    \centering
    \includegraphics[width=0.7\textwidth]{img/insertion sort_lan2/1.png}
    
\end{figure}
\newpage

\begin{itemize}
    \item Chúng ta bắt đầu với phần tử thứ hai của mảng vì phần tử đầu tiên trong mảng được coi là đã được sắp xếp.
    \begin{figure}[H]
        \centering
        \includegraphics[width=0.7\textwidth]{img/insertion sort_lan2/2.png}

    \end{figure}
    
\end{itemize}


\begin{itemize}
    \item So sánh phần tử thứ hai với phần tử thứ nhất và kiểm tra xem phần tử thứ hai có nhỏ hơn không, nếu có thì hoán đổi chúng.

    \begin{figure}[H]
        \centering
        \includegraphics[width=0.7\textwidth]{img/insertion sort_lan2/3.png}
\caption{kết quả việc so sánh và hoán vị.}
    \end{figure}
    
\end{itemize}

\begin{itemize}
    \item Di chuyển đến phần tử thứ ba và so sánh nó với hai phần tử đầu tiên và đặt vào đúng vị trí của nó sao cho giữ đúng tính chất sắp xếp
    \begin{figure}[H]
        \centering
        \includegraphics[width=0.7\textwidth]{img/insertion sort_lan2/4.png}
        \includegraphics[width=0.7\textwidth]{img/insertion sort_lan2/5.png}
         \caption{kết quả sau khi chèn đúng vị trí.}
    \end{figure}
    
\end{itemize}

\newpage

\begin{itemize}
\item Tiếp tục, cứ lấy phần tử kế tiếp chèn vào đúng vị trí trong mảng con màu xanh phía trước. Lặp lại cho đến khi toàn bộ mảng được sắp xếp
    \begin{figure}[H]
        \centering
        \includegraphics[width=0.7\textwidth]{img/insertion sort_lan2/6.png}
        \includegraphics[width=0.7\textwidth]{img/insertion sort_lan2/7.png}
        \includegraphics[width=0.7\textwidth]{img/insertion sort_lan2/8.png}
        \includegraphics[width=0.7\textwidth]{img/insertion sort_lan2/9.png}
        
    \end{figure}
    
\end{itemize}

\begin{figure}[H]
    \includegraphics[width=0.7\textwidth]{img/insertion sort_lan2/10.png}
    \includegraphics[width=0.7\textwidth]{img/insertion sort_lan2/11.png}
    \includegraphics[width=0.7\textwidth]{img/insertion sort_lan2/12.png}
    \includegraphics[width=0.7\textwidth]{img/insertion sort_lan2/13.png}

\end{figure}




\subsubsection{Độ phức tạp}
Thuật toán Insertion sort có độ phức tạp thời gian như sau:
\begin{itemize}
    \item Trường hợp tốt nhất: $O(n)$ (mảng đã sắp xếp).
    \item Trường hợp trung bình: $O(n^2)$.
    \item Trường hợp xấu nhất: $O(n^2)$ (mảng sắp xếp ngược).
\end{itemize}