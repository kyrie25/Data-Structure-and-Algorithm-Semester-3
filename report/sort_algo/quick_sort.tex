\subsubsection{Ý tưởng}
Quick Sort là thuật toán hoạt động bằng cách chia mảng thành hai phần dựa vào pivot (phần tử chốt) được chọn. Một bên bao gồm các giá trị nhỏ hơn pivot, bên còn lại chứa các giá trị lớn hơn pivot. Quick Sort sẽ được gọi đệ quy trên từng phần cho đến khi không thể chia nhỏ hơn nữa, sau đó mảng sẽ được sắp xếp.\cite{fasha2021comparative}

\subsubsection{Mã giả}
\begin{algorithm}[H]
\caption{Quick Sort (Choose middle element as pivot)}
\begin{algorithmic}[1]
\Procedure{QuickSort}{$arr, n$}
    \State \textbf{Input:} Mảng $arr$ gồm $n$ phần tử
    \State \textbf{Output:} Mảng $arr$ được sắp xếp
    \State \Call{QuickSortRecursive}{$A, 0, n - 1$}
\EndProcedure

\Procedure{QuickSortRecursive}{$arr, left, right$}
    \If{$left \geq right$}
        \State \textbf{return}
    \EndIf
    \State $pivot \gets$ \Call{Partition}{$arr, left, right$}
    \State \Call{QuickSortRecursive}{$arr, left, pivot - 1$}
    \State \Call{QuickSortRecursive}{$arr, pivot, right$}
\EndProcedure

\Procedure{Partition}{$arr, left, right$}
    \State $pivot \gets arr[\frac{left + right}{2}]$
    \State $i \gets left, j \gets right$

    \While{$i \leq j$}
        \While{$arr[i] < pivot$}
            \State $i \gets i + 1$
        \EndWhile
    
        \While{$arr[j] > pivot$}
            \State $j \gets j - 1$
        \EndWhile
    
        \If{$i \leq j$}
            \State \textbf{swap} $arr[i]$ \textbf{and} $arr[j]$
            \State $i \gets i + 1$
            \State $j \gets j - 1$
        \EndIf
    \EndWhile
    \State \textbf{return} $i$
\EndProcedure
\end{algorithmic}
\end{algorithm}

\subsubsection{Ví dụ}
Dưới đây là các bước chạy tay của thuật toán \textit{Quick Sort} (với $pivot$ là phần tử giữa mảng) với mảng $[42, 17, 93, 58, 21, 76, 34]$:

\begin{figure}[H]
    \centering
    \includegraphics[width=0.75\linewidth]{img/quick_sort/1.png}
\end{figure}

\begin{enumerate}
    \item Trước tiên ta chọn $pivot = 58$ (phần tử ở giữa mảng) và tiến hành phân hoạch mảng thành hai phần:
    \begin{itemize}
        \item Phần bên trái chứa các phần tử nhỏ hơn hoặc bằng $58$: $[42, 17, 34, 21]$
        \item Phần bên phải chứa các phần tử lớn hơn $58$: $[93, 76]$
    \end{itemize}
    \item Ta sẽ di chuyển các phần tử sao cho các phần tử nhỏ hơn hoặc bằng $pivot$ sẽ nằm bên trái $pivot$ và các phần tử lớn hơn $pivot$ sẽ nằm bên phải $pivot$.
    
    \begin{figure}[H]
        \centering
        \includegraphics[width=0.75\linewidth]{img/quick_sort/2.png}
        
        \includegraphics[width=0.75\linewidth]{img/quick_sort/3.png}
        
    \end{figure}
    
    \begin{figure}[H]
        \centering
        \includegraphics[width=0.75\linewidth]{img/quick_sort/4.png}
    
        \includegraphics[width=0.75\linewidth]{img/quick_sort/5.png}
    
        \includegraphics[width=0.75\linewidth]{img/quick_sort/6.png}
        
        \includegraphics[width=0.75\linewidth]{img/quick_sort/7.png}
        
        \includegraphics[width=0.75\linewidth]{img/quick_sort/8.png}
    \end{figure}

    \item Đến đây ta gọi đệ quy cho hai phần mảng mới được phân hoạch là $[42, 17, 34, 21]$, $[58, 76, 93]$ và cứ tiếp tục như vậy cho đến khi ta không thể phân hoạch nhỏ hơn được nữa.
    \begin{figure}[H]
        \centering
        \includegraphics[width=1\linewidth]{img/quick_sort/9.png}
        
    \end{figure}

    \item Sau khi thực hiện tất cả bước trên ta sẽ thu được mảng đã được sắp xếp.
    \begin{figure}[H]
        \centering
        \includegraphics[width=0.75\linewidth]{img/quick_sort/10.png}
    \end{figure}
\end{enumerate}

\subsubsection{Độ phức tạp}
\begin{itemize}
    \item[\textbf{--}]Độ phức tạp về thời gian:
        \begin{itemize}
            \item[$\bullet$] \textbf{Best Case:} $\mathcal{O}(n \cdot \log{}n)$ 
            \item[$\bullet$] \textbf{Average Case:}  $\mathcal{O}(n \cdot \log{}n)$
            \item[$\bullet$] \textbf{Worst Case:}  $\mathcal{O}(n^{2})$
        \end{itemize}
    \item[\textbf{--}]Độ phức tạp về không gian: $\mathcal{O}(\log{n})$ 
    \item[\textbf{--}]Tính ổn định: Không ổn định
\end{itemize}

